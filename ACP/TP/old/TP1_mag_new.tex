\documentclass[11pt]{article}
\usepackage{times}
\usepackage[latin1]{inputenc}
\usepackage{amsmath}
\usepackage{array}
\usepackage{multirow}
%\usepackage{rotating}
\usepackage{pstricks}

% test de contrôle continu deug mass 2003


\textwidth 15cm
\textheight 24cm
%\topmargin -0.5cm
\oddsidemargin = 0cm %  -0.7cm
\baselineskip 4.5ex
\voffset=-2cm

\def\Indi{\mbox{{\rm 1\kern-.32em I}}}

\parindent=0pt

\pagestyle{empty}

\begin{document}



\vskip -1.5cm


{\begin{tabular}{cc}
\it  Magistère d'Economiste Statisticien\\
\end{tabular}\hfill
{\begin{tabular}{c}
TSE - Université Toulouse 1 Capitole\\
%2008-2009\\
\end{tabular}

\vspace{1cm}

\centerline{\Large\bf  Analyse des données}

\begin{center}
{\Large\bf TP 1 :  Analyse en composantes principales (ACP)  \\  \vspace{0.3cm} avec R  (packages FactoMineR et Factoshiny, fonction PCA())  }
\end{center}

\noindent w
}


\begin{enumerate}
\item Importer le fichier {\tt pays-eu.txt} sous {\tt R}.
\item Réaliser une étude univariée rapide des données à l'aide d'indicateurs numériques et de graphiques.
\item   Donner la matrice des corrélations et les nuages de points associés. Commenter.
\item L'ACP vous paraît-elle justifiée~?
\item Déterminer  le nombre de composantes principales à retenir pour cette ACP. Justifier votre réponse.
\item Interpréter les composantes principales retenues à l'aide des
variables initiales. Donner un (des) graphique(s) permettant de visualiser l'interprétation.
\item Quels sont les pays bien représentés sur chacun des axes retenus~?
Justifier votre réponse.
\item Commenter les contributions des pays aux premier axe.
%\item Donner les corrélations de la variable LARCENRY avec les axes retenus. Commenter.
\item Réaliser le(s) graphique(s) des pays et commenter l'ACP.
\end{enumerate}

\bigskip
{\bf Annexe : Initiation à R }
\bigskip

R utilise des {\bf fonctions} ou des opérateurs qui agissent sur des objets (vecteurs, matrices, data-frames etc.).

\bigskip

{\bf Importation  de données}

Lecture d'un fichier texte~: fonction read.table(.)

\begin{verbatim}
employes=read.table("employes.txt",header=T)
employes[,-1]#enlève la première colonne
head(employes)# début du fichier
employes[,c(2,3,4)] # donne les colonnes 2, 3 et 4 du fichier
employes[1:10,]# donne les 10 premières lignes du fichier
\end{verbatim}

employes est un data-frame (format par défaut sous R,
format obtenu par la lecture de fichiers externes).
\bigskip

{\bf Les jeux de données de R}

R contient plusieurs jeux de données, qui peuvent être chargés par la fonction data.
Pour voir leur liste taper ~: data()

\bigskip
\begin{verbatim}
data(Orange)  # charge le jeu de données Orange
help(Orange)   # donne des informations sur le jeu de données
\end{verbatim}

\bigskip
{\bf Edition intéractive des données}

La fonction edit permet de modifier les données avec la souris. En cliquant sur le nom de la colonne, on peut aussi modifier son type
(real=numérique=quantitative, character=facteur=qualitative).


\begin{verbatim}
data(airquality) # charge le jeu de données airquality
aq=edit(airquality) # édite le jeu de données et le stocke dans aq
\end{verbatim}


Ainsi, le jeu de données initial airquality n'est pas modifié. On peut également utiliser la fonction fix, mais les modifications  qu'on apporte
écrasent le tableau originel.

Pour créer un jeu de données en entrant les données au clavier~:

\begin{verbatim}
donnees=data.frame()  # crée le tableau donnees
fix(donnees)  # édite le tableau vide
\end{verbatim}

%\newpage

{\bf Résumé numérique des variables}

Les fonctions suivantes donnent les statistiques descriptives usuelles pour les variables quantitatives~:

\begin{verbatim}
attach(airquality) # rend les variables appelables directement
mean(Ozone) # moyenne de Ozone
sd(Ozone)  # écart-type
var(Ozone)  # variance
median(Ozone) # médiane
quantile(Ozone)  # donne le min, le max et les quartiles
summary(Ozone)  # min, max, moyenne et quartiles
summary(airquality)  # statistiques pour le data frame tout entier
cor(Ozone,Temp) # corrélation entre Ozone et Temp

\end{verbatim}


Si on veut que R calcule ces statistiques en présence de données manquantes, il faut rajouter l'argument na.rm=T aux fonctions ci-dessus.

Pour avoir le nombre de données non manquantes~: sum(!is.na(Ozone))

%{\bf{\it Application}~: Calculer les différentes statistiques simples pour les variables quantitatives du fichier employes.}

\bigskip

Si la variable est de type facteur, summary donne simplement les effectifs des différentes modalités, ce que l'on peut aussi obtenir par la fonction table.

\bigskip

Comme cela n'a pas de sens de calculer une moyenne ou une variance pour une variable qualitative, toute
variable qualitative codée en numérique doit être mise au type facteur (fonction factor).
\begin{verbatim}
summary(employes)# noter ce qu'il fait pour la variable sexe
employes$sexe=factor(employes$sexe,labels=c("F","M"))#modifie le type
de la variable sexe dans le tableau
#cette opération se fait également en éditant le fichier
# et en changeant son type
summary(employes)
table(employes$sexe) #fait la même chose que summary pour une qualitative
round(prop.table(table(employes$stat_pro)),digits=2)#tableau des fréquences
relatives arrondies à 2 décimales

#Tableau de contigence
employes$stat_pro=factor(employes$stat_pro, labels=c("employé de bureau",
+   "agent de sécurité","manager")) # transforme la variable qualitative
 codée stat_pro en facteur
 tab= table(employes$sexe,employes$stat_pro) # crée le tableau de contingence et
le stocke dans tab

\end{verbatim}
%\newpage
{\bf Graphiques}

{\it Pour les variables qualitatives~:}

Diagramme circulaire~: pie

\begin{verbatim}
attach(employes)
table(stat_pro)
pie(table(stat_pro))
barplot(prop.table(table(stat_pro)),col=1:3)
\end{verbatim}
%\newpage
Diagramme en colonnes

\begin{verbatim}
barplot(table(stat_pro))
barplot(prop.table(table(stat_pro)),col=1:3)
\end{verbatim}

{\it Pour les variables quantitatives discrètes~:}

Diagramme en bâtons (respecte l'espacement des valeurs)~:

\begin{verbatim}
#diagramme en effectif
plot(table(educ),lwd=5,col="red",xlab="Nombre d'années d'études",
ylab="effectif",main="Diagramme en bâtons de la variable educ")

# diagramme en fréquence
n=length(educ)
plot(table(educ)/n,lwd=5,col="red",xlab="Nombre d'années d'études",
ylab="effectif",
main="Diagramme en bâtons de la variable educ")
\end{verbatim}

{\it Pour les variables quantitatives continues~:}

Histogramme~: \begin{verbatim}  hist(salaire)  \end{verbatim}

\bigskip

Boîte à moustaches~: \begin{verbatim} boxplot(salaire)  \end{verbatim}

\bigskip

Diagramme de dispersion (nuage de points)
\begin{verbatim}
 plot(salaire,age,xlab="âge",ylab="salaire")
 # salaire en ordonnée, âge en abscisse
 avec les labels des axes
\end{verbatim}

\medskip

{\it Exercice (à faire en autonomie)}

\medskip
Le fichier de données {\tt employes.txt} concerne 474 employés d'une entreprise américaine.
Les variables relevées sont les suivantes~:
\begin{itemize}
\item sexe (2 pour masculin, 1 pour féminin)
\item éducation~: nombre d'années passées à l'école
\item statut professionnel (1 si "employé de bureau", 2 si "agent de sécurité", 3 si "manager")
\item salaire annuel à l'embauche dans l'entreprise (en dollars)
\item salaire annuel courant (en dollars)
\item ancienneté dans l'entreprise en nombre de mois
\item expérience passée (en nombre de mois)
\item nationalité (1 pour américaine, 0 sinon)
\item âge (en années)
\end{itemize}
\begin{enumerate}
\item Importer le fichier {\tt employes.txt} sous {\tt R}.
\item Transformer les variables qualitatives du fichier en facteurs et mettre des labels.

\item Donner des résumés numériques et des graphiques représentant les distributions des variables  du fichier "employes".

\item Donner la matrice des corrélations (arrondie à 2 décimales) entre les variables quantitatives du fichier. Commenter.


\item Sauver le script.
\end{enumerate}


\end{document} 